% Created 2018-09-11 mar 11:00
% Intended LaTeX compiler: pdflatex
\documentclass[11pt]{article}
\usepackage[utf8]{inputenc}
\usepackage[T1]{fontenc}
\usepackage{graphicx}
\usepackage{grffile}
\usepackage{longtable}
\usepackage{wrapfig}
\usepackage{rotating}
\usepackage[normalem]{ulem}
\usepackage{amsmath}
\usepackage{textcomp}
\usepackage{amssymb}
\usepackage{capt-of}
\usepackage{hyperref}
\usepackage[spanish]{babel}
\usepackage{fancyvrb}
\author{Dr. Miguel Carrillo Barajas \\
Estefanía Prieto Larios \\
Mauricio Esquivel Reyes \\
}
\date{}
\title{Práctica 01 \\
Lógica Computacional \\
Universidad Nacional Autónoma de México}
\hypersetup{
 pdfauthor={Dr. Miguel Carrillo Barajas \\
Estefanía Prieto Larios \\
Mauricio Esquivel Reyes \\
},
 pdftitle={Práctica 01 \\
Lógica Computacional \\
Universidad Nacional Autónoma de México},
 pdfkeywords={},
 pdfsubject={},
 pdfcreator={Emacs 25.3.2 (Org mode 9.1.14)}, 
 pdflang={Spanish}}
\begin{document}

\maketitle
\section{Estructuras}
\label{sec:org43fe393}
\subsection{Naturales}
\label{sec:org90c0d76}
Consideremos la siguiente representación de los números naturales

\begin{verbatim}
data Natural = Cero | Suc Natural deriving (Eq, Show)
\end{verbatim}
\subsubsection{mayorQue :: Natural -> Natural -> Bool}
\label{sec:org40f8f56}
Dados dos naturales nos dice si el primero es mayor que el segundo.\\
Ejemlos:
\begin{itemize}
\item Main> mayorQue Cero (Suc Cero)\\
False
\item Main> mayorQue (Suc Cero) Cero\\
True
\end{itemize}
\subsubsection{menorQue :: Natural -> Natural -> Bool}
\label{sec:org9ccf5f7}
Dados dos naturales nos dice si el primero es menor que el segundo.\\
Ejemlos:
\begin{itemize}
\item Main> menorQue Cero (Suc Cero)\\
True
\item Main> menorQue (Suc Cero) Cero\\
False
\end{itemize}
\subsubsection{igual :: Natural -> Natural -> Bool}
\label{sec:orgf7f5429}
Dados dos naturales nos dice si son iguales.\\
Ejemplos:
\begin{itemize}
\item Main> igual Cero (Suc Cero)\\
False
\item Main> igual (Suc Cero) (Suc Cero)\\
True
\end{itemize}
\subsection{Lista de naturales}
\label{sec:org6026b9a}
Consideremos la siguiente definición de las listas de naturales.
\begin{verbatim}
data ListaDeNaturales = Nil | Cons Natural ListaDeNaturales
\end{verbatim}
\subsubsection{concate :: ListaDeNaturales -> ListaDeNaturales -> ListaDeNaturales}
\label{sec:org230b4d5}
Dadas dos listas de naturales regresar la concatenación de ambas.\\
Ejemplos:
\begin{itemize}
\item Main> concate (Cons (Suc Cero) Nil) (Cons Cero (Cons (Suc (Suc Cero)) Nil))\\
Cons (Suc Cero) (Cons Cero (Cons (Suc (Suc Cero)) Nil))
\item Main> concate (Cons Cero (Cons (Suc (Suc Cero)) Nil)) (Cons (Suc Cero) Nil)\\
Cons Cero (Cons (Suc (Suc Cero)) (Cons (Suc Cero) Nil))
\end{itemize}
\subsubsection{reversa :: ListaDeNaturales -> ListaDeNaturales}
\label{sec:orgc3617cf}
Dada una lista regresar la reversa de dicha lista.\\
Ejemplos:
\begin{itemize}
\item Main> reversa (Cons Cero (Cons (Suc (Suc Cero)) (Cons (Suc Cero) Nil)))\\
Cons (Suc Cero) (Cons (Suc (Suc Cero)) (Cons Cero Nil))
\end{itemize}
\section{Lógica Proposicional}
\label{sec:org2b24091}
Consideremos la siguiente representación de la lógica proposicional.
\begin{verbatim}
-- Tipo de dato indice
type Indice = Int

-- Tipo de dato fórmula
data PL = Top | Bot  | Var Indice
              | Oneg PL 
              | Oand PL PL | Oor PL PL 
              | Oimp PL PL deriving (Eq, Show)
\end{verbatim}
\subsection{Conjunciones}
\label{sec:orga60673c}
\subsubsection{conj :: PL -> [PL]}
\label{sec:org50aaf73}
Dada una fórmula regresar una lista con las conjunciones de dicha fórmula.\\
Ejemplo: 
\begin{itemize}
\item Main> conj Oor (Oand (Var 1) Oneg \$ Var 2) (Oand Bot (Var 3))\\
[Oand (Var 1) Oneg \$ Var 2, Oand Bot (Var 3)]
\end{itemize}
\subsubsection{numConj :: PL -> Int}
\label{sec:org9cc558c}
Dada una fórmula regresar el número de conjunciones que tiene dicha fórmula.\\
Ejemplo:
\begin{itemize}
\item Main> conj Oor (Oand (Var 1) Oneg \$ Var 2) (Oand Bot (Var 3))\\
2
\end{itemize}
\subsection{Semántica}
\label{sec:orgb4af71c}
Consideremos los tipos de datos Valuación y Modelo de la forma 
\begin{verbatim}
type Valuacion = Indice -> Bool
type Modelo = [Indice]
\end{verbatim}
\subsubsection{satMod :: Modelo -> PL -> Bool}
\label{sec:org76761ac}
Dada una fórmula proposicional, evalúar la fórmula de acuerdo a un modelo.\\
Ejemplos:
\begin{itemize}
\item Main> satMod [1] (Oneg (Var 1))\\
False
\item Main> satMod [] (Oimp (Var 1) (Var 2))\\
True
\end{itemize}
\subsubsection{satPL :: Valuacion -> PL -> Bool}
\label{sec:orga272c75}
Dada una fórmula proposicional, evalúar la fórmula de acuerdo a una valuación.\\
(Nota: en clase realizamos la función modeloToValuacion)
\subsection{Formas normales}
\label{sec:org9df0563}
\subsubsection{esClausula :: PL -> Bool}
\label{sec:org344e232}
Dada una fórmula nos indica si es una clausula.
\subsubsection{esCNF :: PL -> Bool}
\label{sec:orgfc834b5}
Dada una fórmula nos indica si esta en forma normal de conjunción.
\subsubsection{esTermino :: PL -> Bool}
\label{sec:org8661a2d}
Dada una fórmula nos indica si es un término.
\subsubsection{esDNF :: PL -> Bool}
\label{sec:org133e3a2}
Dada una fórmula nos indica si esta en forma normal de disyunción.
\subsubsection{toCNF :: PL -> PL}
\label{sec:org2f172dd}
Dada una fórmula en NNF y sin implicaciones, dar su CNF, tal que sean logicamente equivalentes.
\end{document}
